\documentclass{beamer}
\usepackage{ucs}
\usepackage[utf8]{inputenc}
\usepackage{ngerman}
\usepackage[ngerman]{babel}
\usepackage{floatflt}
\usepackage{listings}
\definecolor{lightgray}{rgb}{.95,.95,.95}
\definecolor{darkblue}{rgb}{0,0,.6}
\definecolor{darkred}{rgb}{.6,0,0}
\definecolor{darkgreen}{rgb}{0,.6,0}
\definecolor{red}{rgb}{.98,0,0}
\lstloadlanguages{C++}
\lstset{%
	language=C++,
	basicstyle=\footnotesize\ttfamily,
	commentstyle=\color{darkgreen},%\itshape
	keywordstyle=\bfseries\color{darkblue},
	stringstyle=\color{darkred},
	showspaces=false,
	showtabs=false,
	columns=fixed,
	numbers=left,
	frame=none,
	numberstyle=\tiny,
	breaklines=true,
%	backgroundcolor=\color{lightgray},
	showstringspaces=false,
	xleftmargin=1cm
}%

\usepackage{beamerthemesplit}

\title{LIFI Live Feedback}
\subtitle{Meilenstein 3}
\author{Gruppe 5 (Paul Walger, Sebastian Götte)}
\institute{TU Berlin}
\date{17. Juli 2015}
\usetheme{Copenhagen}
\usecolortheme{crane}

\begin{document}

\frame{\titlepage}

\section{Backend}
% Datenmodell
% Abstimmungssicherheitsgedöns
% Rollenmodell
% Flask
% Auth-Framework
% Testing-Framework und Api-Doc
\begin{frame}
    \frametitle{This is a title}
    \begin{itemize}
        \item Something
        \item Something else
        \item Example
    \end{itemize}
\end{frame}

\section{Web-Frontend}
% Screenshots!
\section{Android-Frontend}
% Screenshots!
% Unterschiede zum Web-Frontend
\section{Entwicklung und Deployment}
% Entwicklung mit SVN/GIT
% Deployment via Ansible
% Testen mit Vagrant
% Lokales Testen mit Virtualenv und Gulp
\section{Demo}
\end{document}
